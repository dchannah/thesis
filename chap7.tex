\chapter{Concluding Remarks}

\section{Dissertation Summary}

This dissertation explores two major themes in nanoscience: Heat generation/transport and light emission from nanostructured Si. Chapters 2, 3, and 4 address the former issue. In chapter 2, we examine the early stages of heat generation - that is, the conversion of carrier energy to lattice heat. We utilize a recently developed spectroscopic method, femtosecond stimulated Raman spectroscopy (FSRS), which is unique in its ability to resolve low-frequency vibrational populations with subpicosecond time resolution. We are able to directly measure, for the first time, carrier-cooling by phonon emission in CdSe nanocrystals (NCs). We find that electrons cool primarily \emph{via} resonant, Auger-like energy transfer to holes, which cool by emitting phonons. These results demonstrate the utility of FSRS in quantitatively measuring the dissipation of excitonic energy into other degrees of freedom and confirm as-yet unproven theories regarding electron-to-hole energy transfer. In chapter 2 we also present theoretical modeling which provides a mechanism for the slowdown of electron-phonon thermalization in small plasmonic nanoparticles upon thiol-passivation. Considering the growing utility of plasmonic nanostructures in generating hot electrons for e.g. catalysis, these results should provide guidance to researchers hoping to manipulate hot electron lifetimes. \par
Chapter 3 addresses issues related to nanoscale thermal transport. First, we present our discovery of a dynamic optical signature of acoustic phonon transport out of semiconductor NCs. By measuring the size-dependent rate of this signature we are able to demonstrate that heat transport of matrix-embedded NCs occurs on a 10-100 ps time scale in a diffusion-limited fashion. We also utilize MD simulations to demonstrate that in addition to surface ligands mediating interfacial thermal conductance, the structure of the underlying semiconductor surface plays an important role by determining the grafting density of these ligands. We expand on this knowledge in chapter 4, which focuses on making structural modifications to NCs to manipulate thermal processes. Here, we utilize the aforementioned optical signature of thermal outflow to demonstrate that the growth of a wide-bandgap semiconductor shell permits independent tuning of NC optical and thermal properties. Furthermore, we present theoretical results which suggest that passivation of NC surfaces with small, inorganic ligands or covalently bound species may provide similar thermal benefits but without confining charge carriers to the NC core. \par
Chapters 5 and 6 focus on group-IV NCs. In chapter 5, we examine long-lived PL (PL) from Si NCs. Using pressure-dependent PL spectroscopy along with combined quantum-classical modeling, we definitively assign red PL to band-edge emission from NC core states, which exhibits indirect-gap character despite substantial quantum confinement. For the first time, we also spectrally and temporally resolve high-energy PL and associate it with a persistent amorphous surface layer, a conclusion we support using TEM, Raman, and MD simulations. \par
In chapter 6, we explore interesting structural properties of Si and Ge NCs. We report XRD characterizations of these NCs at pressures ranging from 0-100 GPa. As has been observed in other NCs, both Si and Ge exhibit elevated phase transition pressures relative to the bulk phase. Interestingly, in stark constrast to binary semiconductor NC compositions, we note compressibility values matching the bulk phase for both Si and Ge. We explore the possible origins of this behavior using MD simulations, with a focus on the role played by the surface.

\section{Future Directions}

\subsection{Heat Generation/Transport}
Chapters 2-4 have provided a much clearer picture of the processes and parameters governing heat generation and transport in semiconductor NCs. They also present results which demonstrate that structural modifications affect these processes in a controllable way. A promising next step for work in this area is to explore the role of dimensionality. For example, do phonon population dynamics differ significantly in nanorods (1D) or nanoplates (2D)? Samples exhibiting multiple observable vibrational modes are also ripe for study: By correlating population transfer between modes having a known spatial separation, one could answer open questions regarding the physics of heat transport at nanometer length scales. Another avenue regarding dimensionality is suggested by the simulations in chapter 3 - might it be possible to engineer directional heat flow by synthesizing structures which preferentially expose high-thermal-conductance surfaces? Finally, thermostatting and barostatting algorithms have recently been developed which enable the application of non-equilibrium MD simulations to non-periodic systems, such as an isolated nanoparticle in a solvent droplet. We are currently applying these methods to colloidal NC structures to obtain a fully atomistic, theoretical picture of heat transport in such systems.

\subsection{Light Generation}
While chapter 5 thoroughly explores the origin of PL in plasma-synthesized Si NCs, it will be important to examine whether other synthetic routes yield particles presenting an (emissive) amorphous surface layer. We are also currently investigating the origin of PL in Ge NCs. While Si NC PL was shown to be dominated by an indirect-gap transition, one might expect quantum confinement to more strongly mix the direct- and indirect-gap transitions in Ge, which are far more energetically proximal than in Si. \par
Finally, the interesting pressure-dependent structural behavior of group-IV NCs prompts several open questions. Why does compressibility of group-IV NCs match the bulk phase while binary NC compositions differ significantly from bulk in this regard? What is the exact role of the surface in mediating the structural response of NCs to applied pressure? Answering these questions may enable surface chemistry-based approaches to tuning the mechanical properties of nanostructures. While we have made some progress toward answering these questions using MD simulations and present this work in chapter 6, there are many avenues open for further exploration. In particular, efforts are underway to utilize transition path sampling algorithms with the aim of more realistically simulating NC phase transitions. 